% El informe de práctica de estudios debe presentarse
% anillado en tapas transparentes, hojas TAMAÑO CARTA
% y escrito en letra tamaño 12, times new roman,  a
% espacio sencillo, y en tercera persona.

% La información escrita podrá complementarse con dibujos
% y fotografías, tablas, gráficos, análisis matemático y
% citas bibliográficas, en la medida discreta y propia
% para la más fácil comprensión de la materia estudiada.


\documentclass[a4paper,12pt,oneside]{report}

\usepackage[utf8]{inputenc}
\usepackage{graphicx}
\usepackage[right=2cm,left=3cm,top=2cm,bottom=2.2cm]{geometry}
\usepackage[font=footnotesize]{caption}
\usepackage[spanish]{babel}
\usepackage{wrapfig}
\usepackage{fancyhdr}
\usepackage{subfig}
\usepackage[colorlinks=true,linkcolor=blue,urlcolor=blue]{hyperref}
\usepackage{moreverb}
\usepackage{listings}
\pagestyle{fancy}
\usepackage{amsmath, amsthm, amssymb}
\usepackage[only,mapsfrom,Mapsto,Mapsfrom]{stmaryrd}

\usepackage[absolute]{textpos} %PARA PONER UNA IMAGEN EN PANTALLA COMPLETA, PORTADA ETC


\usepackage{color}


%\RequirePackage{eurosym}
%\RequirePackage[style=long,cols=2,border=none,toc=true]{glossary}
%\RequirePackage{makeidx}
%\RequirePackage{supertabular}
\usepackage{listings}
% Definiendo colores para los listados de código fuente
\definecolor{violet}{rgb}{0.5,0,0.5}
\definecolor{navy}{rgb}{0,0,0.5}

%\definecolor{hellgelb}{rgb}{1,1,0.8}
\definecolor{verdeclaro}{RGB}{221,248,204}
\definecolor{rojizo}{RGB}{247,112,57}
\definecolor{plomo}{RGB}{238,238,238}
\definecolor{celeste}{RGB}{228,247,255}
\definecolor{colKeys}{rgb}{0,0,1}
\definecolor{colIdentifier}{rgb}{0,0,0}
\definecolor{colComments}{rgb}{1,0,0}
\definecolor{colString}{rgb}{0,0.5,0}


\lstset{
        float=hbp,
        basicstyle=\ttfamily\small,
        identifierstyle=\color{colIdentifier},
        keywordstyle=\color{colKeys},
        stringstyle=\color{colString},
        commentstyle=\color{colComments},
        columns=flexible,
        tabsize=4,
        frame=single,
        extendedchars=true,
        showspaces=false,
        showstringspaces=false,
        numbers=left,
        numberstyle=\tiny,
        breaklines=true,
        backgroundcolor=\color{verdeclaro},
        breakautoindent=true,
        captionpos=b
}

\newcommand{\codigofuente}[3]{%
  \lstinputlisting[language=#1,caption={#2}]{#3}%
}

%-------------Formato de títulos de capítulos -------------------------

% http://zoonek.free.fr/LaTeX/LaTeX_samples_chapter/0.html
% titulo #27

\makeatletter
\def\thickhrulefill{\leavevmode \leaders \hrule height 1ex \hfill \kern \z@}
\def\@makechapterhead#1{%
  \vspace*{10\p@}%
  {\parindent \z@
    {\raggedleft \reset@font
      \scshape \@chapapp{} \thechapter\par\nobreak}%
    \par\nobreak
    \vspace*{30\p@}
    \interlinepenalty\@M
    {\raggedright \Huge \scshape #1}%
    \par\nobreak
    \hrulefill
    \par\nobreak
    \vskip 30\p@
  }}
\def\@makeschapterhead#1{%
  \vspace*{10\p@}%
  {\parindent \z@
    {\raggedleft \reset@font
      \scshape \vphantom{\@chapapp{} \thechapter}\par\nobreak}%
    \par\nobreak
    \vspace*{30\p@}
    \interlinepenalty\@M
    {\raggedright \Huge \scshape  #1}%
    \par\nobreak
    \hrulefill
    \par\nobreak
    \vskip 30\p@
  }}


%fin titulo



%-------------Encabezado, pie de páginas y numeración-------------------------

% Cada hoja deberá tener un pie de página que indicará
% el nombre de la empresa o institución donde se realizó
% la práctica desplazado hacia la izquierda y un
% encabezado en el cual se indicará en capítulo
% correspondiente del informe, este último deberá estar
% desplazado hacia la derecha. La letra y el tamaño
% será el mismo que el usado en el pie de página
% (negrita y cursiva).

% Todas las hojas que se encuentran a continuación de
% los resúmenes irán numeradas en la parte inferior
% derecha partiendo de página nº1. La letra será la
% misma del informe tamaño 10 (negrita y cursiva).


% http://aristarco.dnsalias.org/node/21

\fancyhf{} % borrar todos los ajustes

% En lo siguiente, fancyhead sirve para configurar la cabecera, fancyfoot para el pie.
% Justificación: C=centered, R=right, L=left, (nada)=LRC
% Página: O=odd, E=even, (nada)=OE



\fancyhead[R]{\footnotesize \leftmark}
\fancyfoot[L]{\footnotesize \scshape CEIS-UFRO.}
\fancyfoot[R]{\small \thepage}
\renewcommand{\chaptermark}[1]{\markboth{\scshape #1}{}}



% aplicar estilo fancy al inicio de un capítulo
\fancypagestyle{plain}{
   \fancyhead{} % elimina cabeceras en páginas "plain"
   \renewcommand{\headrulewidth}{0pt} % así como la raya
}




\begin{document}

% Portada -----------------------------------------------------------------------


%\begin{textblock*}{297mm}(-6mm,0mm)
%%\includegraphics[height=297mm,width=210mm]{images/fondoPortada.png}
%\end{textblock*}
%\pagebreak
\thispagestyle{empty} %quita la numeración de las paginas
\title{\color{rojizo}{\textbf{Instructivo instalaci\'on: ``Maqueta Sistema de registro de adolecentes infractores de ley (Fondef D08I1205)''}}} % Título
\author{Departamento de psicolog\'ia - UFRO} % Autor. Pueden ser varios agregando \and Otro autor
\date{\today} % Fecha siempre actualizada al día presente al compilar.
\maketitle
\newpage




\chapter*{Sumario}

El objetivo del presente instructivo es servir de apoyo a la instalaci\'on y configuraci\'on de la maqueta del Sistema de registro de adolecentes infractores de ley (Fondef D08I1205). Se debe tener presente que las configuraciones indicadas de instalaci\'on son requisito para que el sistema pueda funcionar correctamente, por lo cual deben ser seguidas al pie de la letra.\newline

El sistema est\'a basado en el framework Symfony 1.4 por lo que algunos de los pasos que se describen en el proceso corresponden a requisitos heredados de dicho framework, se solicita seguir el procedimiento descrito en las p\'aginas siguientes.\newline





\tableofcontents %genera el indice
\thispagestyle{empty}

%\chapter{Introducción}

\section{Tecnologías involucradas}

Entre las tecnologías utilizadas en el transcurso de este taller se encuentran:
\newline
\newline

%---------CentOS------------------
\begin{wrapfigure}{r}{5cm}
%centering
%Logo CentOS
\includegraphics[width=5cm]{images/centos.jpg}
\caption{CentOS.}
\label {fig:centos}
\vspace{1cm}
\end{wrapfigure}
$ $ 
\vspace{-1.5cm}

\begin{description}
\item CentOS

(Community Enterprise Operating System) es la distribución libre del sistema operativo  Red Hat Enterprise Linux RHEL, compilado por voluntarios a partir del código fuente liberado por Red Hat. CentOS es una distribución enfocada a servidores pero también cuenta con su versión para escritorio. Esta distribución fué utilizada en esta práctica como un requisito impuesto por la empresa.\emph{(figura \ref{fig:centos}).}
\end{description}


%---------GlassFish------------------


\begin{wrapfigure}{r}{5cm}
%centering
%Logo Glassfish
\includegraphics[width=5cm]{images/glassfish_logo.png}
\caption{GlassFish.}
\label {fig:glassfish}

\end{wrapfigure}
$ $ 
\vspace{-0.5cm}
\newline

\begin{description}
\item GlassFish

GlassFish es un servidor de aplicaciones desarrollado por Sun Microsystems que implementa las tecnologías definidas en la plataforma Java2EE y permite ejecutar aplicaciones que siguen esta especificación. Es gratuito y de código libre, se distribuye bajo un licenciamiento dual a través de la licencia CDDL y la GNU GPL.\emph{(figura \ref{fig:glassfish}).}
\end{description}


%---------Liferay------------------

\begin{wrapfigure}{r}{5cm}
%centering
%Logo CentOS
\includegraphics[width=5cm]{images/liferay.jpg}
\caption{Liferay.}
\label {fig:liferay}
\vspace*{0.5cm}
\end{wrapfigure}
$ $ 
\vspace{-0.5cm}


\begin{description}
\item Liferay

Liferay es un portal de gestión de contenidos de código abierto escrito en Java que soporta el estándar portlet. Los portlet son componentes  de las interfaces de usuario las cuales pueden ser gestionadas, configuradas y visualizadas por un portal web, en este caso específico se utilizó Liferay. \emph{(figura \ref{fig:liferay}).}
\end{description}

%---------OpenMRS------------------
\begin{wrapfigure}{r}{5cm}
%centering
%Logo OpenMRS
\includegraphics[width=5cm]{images/openmrs.png}
\caption{OpenMRS.}
\label {fig:openmrs}
\vspace{1cm}
\end{wrapfigure}
$ $ 
\vspace{-0.5cm}

\begin{description}
\item OpenMRS

OpenMRS es una aplicación  bajo licencia GPL programada sobre el lenguaje Java que cumple diferentes funcionalidades de un sistema de registros médicos como: registro de pacientes, registro de especialistas, registro de encuentros entre paciente y especialista, generación de reportes, mantenedor de medicamentos y equipamiento en general.\emph{(figura \ref{fig:openmrs}).}
\end{description}
\newpage 
%Introducción a los Web Containers J2EE
\chapter{Instalación del sistema sobre Ubuntu 10.04.1 LTS}
% \addcontentsline{toc}{chapter}{Trabajo realizado por el alumno}

\section{Configuraciones previas}

\subsection{Configurando  MySQL (my.cnf)}

Lo primero que se debe realizar es \textbf{agregar el soporte para mayúsculas y minúsculas en MySQL}. Para lograrlo, editamos el fichero de configuraci\'on de MySQL (típicamente en ``/etc/mysql/my.cnf'' para sistemas *nix) y en la secci\'on \textbf{mysqld} e incluimos la linea: ``lower\_case\_table\_names = 1'' :

\lstset{language=sh}
\begin{lstlisting}
root@localhost: # /home/knx/public_html/www/FONDEF-D08I1205# nano /etc/mysql/my.cnf
\end{lstlisting}


\lstset{language=sh}
\begin{lstlisting}
  [mysqld]
  lower_case_table_names = 1
  #
  # * Basic Settings
\end{lstlisting}

Guardamos y reiniciamos MySQL:

\lstset{language=sh}
\begin{lstlisting}
/etc/init.d/mysql restart
\end{lstlisting}
\newpage

\subsection{Creaci\'on y configuraci\'on de la BD MySQL (PhpMyAdmin)}

A continuaci\'on se debe crear una base de datos de nombre ``psico'' y cargar el script SQL que se adjunta a este documento. Una vez creada la base de datos de deben modificar los par\'ametros de conecci\'on del sistema como se muestra a continuaci\'on:


\lstset{language=sh}
\begin{lstlisting}
 root@localhost: #  nano config/databases.yml
 -------
 param:
    dsn:      mysql:host=127.0.0.1;dbname=psico
    username: root
    password: holahola
\end{lstlisting}

donde por ejemplo  ``holahola'' representa la clave de la base de datos; asi mismo:  host, dbname,username y password el resto de par\'ametros de conecci\'on.\newline

Adem\'as se debe configurar la base de datos en otros 2 archivos:

\lstset{language=sh}
\begin{lstlisting}
 root@localhost: #  ~/public_html/www/FONDEF-D08I1205$ nano apps/frontend/config/app.yml
 root@localhost: #  ~/public_html/www/FONDEF-D08I1205$ nano apps/backend/config/app.yml

    database:
        db_name: psico
        db_host: 127.0.0.1
        db_user: root
        db_pass: holahola
\end{lstlisting}

Donde se deben configurar los respectivos par\'ametros de connecci\'on.



\subsection{Configurando PHP (php.ini)}

Una de las limitantes de este sistema es la gran complejidad de muchos de sus formularios, los cuales requieren enviar grandes cantidades de informaci\'on cuando son guardados. Es por esto que una de las configuraciones requeridas es aumentar el tiempo de respuesta de los script PHP y el tama\~no de informaci\'on que se puede enviar por medio de POST en el protocolo HTML.\newline

El archivo php.ini contiene los parámetros de configuración de PHP, entre ellas los parámetros relacionados para subir archivos, estas propiedades son:


\begin{enumerate}
 \item post\_max\_size: tamaño máximo de datos enviados por POST.
 \item upload\_max\_filesize: tamaño máximo para subir archivos.
 \item max\_execution\_time: tiempo máximo de ejecución de cada script en segundos.
 \item max\_input\_time: tiempo máximo para analizar la petición de datos.
\end{enumerate}

Para modificar esto se debe editar el fichero de configuraci\'on de php (php.ini) y setear los valores como sigue: \newline

\begin{enumerate}
 \item post\_max\_size: 100M
 \item upload\_max\_filesize: 25M
 \item max\_execution\_time: 900
 \item max\_input\_time: 900
\end{enumerate}

Se edita el archivo con los par\'ametros y luego se reinicia apache para que cargue las nuevas configuraciones.
\lstset{language=sh}
\begin{lstlisting}
  nano /etc/php5/apache2/php.ini
  root@localhost: # /home/knx/public_html/www/FONDEF-D08I1205# /etc/init.d/apache2 restart
  * Restarting web server apache2                                                                                                                                                             apache2:   Could not reliably determine the server's fully qualified domain name, using 127.0.1.1 for ServerName
   ... waiting apache2: Could not reliably determine the server's fully qualified domain name, using 127.0.1.1 for ServerName
                                                                                                                                                                                    [ OK ]
  root@localhost: # /home/knx/public_html/www/FONDEF-D08I1205#
\end{lstlisting}


\subsection{Configurando Virtualhost (a criterio)}

Los scripts que se encuentran en el directorio web/ son los únicos puntos de entrada a la aplicación. Por este motivo, debe configurarse el servidor web para que puedan ser accedidos desde Internet. En el servidor de desarrollo y en los servicios de hosting profesionales, se suele tener acceso a la configuración completa de Apache para poder configurar servidores virtuales (virtual host). En los servicios de alojamiento compartido, lo normal es tener acceso solamente a los archivos .htaccess. Recomendamos la creaci\'on del virtual host, a\'un cuando el administrador de sistemas puede decidir una ofrma alternativa de apuntar el acceso a la web en la carpeta web/.\newline


El siguiente listado muestra un ejemplo de la configuración necesaria para crear un nuevo servidor virtual en Apache mediante la modificación del archivo httpd.conf.\newline



\lstset{language=sh}
\begin{lstlisting}
Listen 127.0.0.1:8081

<VirtualHost 127.0.0.1:8081>
DocumentRoot "/home/knx/public_html/www/FONDEF-D08I1205/web"
DirectoryIndex index.php
<Directory "/home/knx/public_html/www/FONDEF-D08I1205/web">
AllowOverride All
Allow from All
</Directory>
</VirtualHost>
\end{lstlisting}


En la configuración del listado anterior, se debe sustituir la variable sf\_symfony\_data\_dir por la ruta real del directorio de datos de Symfony.\newline

Como alternativa a este procedimiento se puede crear un subdominio como lo har\'ia para cualquier p\'agina, pero redirigirlo a la ruta del sistema, por ejemplo:
\lstset{language=sh}
\begin{lstlisting}
NOMBRESUBDOMINIO.ufro.cl redirigirlo a /home/knx/public_html/www/FONDEF-D08I1205/web
\end{lstlisting}


que es el directorio interno donde se aloja el sistema. %Instalación y Tunning de GlassFish 2 + Liferay sobre Centos 5.3
%Instalación y configuración de OpenMRS y Alfresco.


\section{OpenMRS}

\subsection{Sobre OpenMRS}


\begin{description}
\item OpenMRS

OpenMRS es una aplicación  bajo licencia GPL programada sobre el lenguaje Java que cumple diferentes funcionalidades de un sistema de registros médicos como: registro de pacientes, registro de especialistas, registro de encuentros entre paciente y especialista, generación de reportes, mantenedor de medicamentos y equipamiento en general.
\end{description}


\subsection{Obtener OpenMRS}

Se ingresa al sitio de descarga de OpenMRS:
\begin{center}\textbf{http://openmrs.org/wiki/Downloads}\end{center}

En la sección \textit{Official Releases} seleccionamos el fichero openmrs.war correspondiente a la última versión disponible para este caso esta corresponde a la versión 1.6.1.
\newline

Esto nos llevará a la siguiente URL en la cual debemos aceptar la licencia de OpenMRS antes de descargar.Utilziando esta URL realizamos la descarga.
\newline

\lstset{language=sh}
\begin{lstlisting}
root@localhost:  wget -b http://resources.openmrs.org/builds/releases/OpenMRS_1.6.1/openmrs.war
\end{lstlisting}

\subsection{Agregar como aplicación a GlassFish}

Se ingresa  al panel de administración de GlassFish en http://miserver.lazos.cl:4848/  \emph{(figura \ref{fig:panelglassfish})}.\newline

\begin{figure}
 \centering
 \includegraphics[width=15cm]{./images/panelglassfish.png}
 % panelglassfish.png: 1024x768 pixel, 72dpi, 36.12x27.09 cm, bb=0 0 1024 768
 \caption{Vista del ingreso al panel de administración de GlassFish}
 \label{fig:panelglassfish}
\end{figure}


Luego utilizando el nombre y password de usuario admin se ingresa  a Applications, Web Applications y click  en deploy. \emph{(figura \ref{fig:deployglassfish})}.\newline

\begin{figure}
 \centering
 \includegraphics[width=15cm]{./images/deployglassfish.png}
 % deployglassfish.png: 796x436 pixel, 72dpi, 28.08x15.38 cm, bb=0 0 796 436
 \caption{Vista del panel de administración en la pantalla deploy de GlassFish}
 \label{fig:deployglassfish}
\end{figure}

Se selecciona el fichero que contiene OpenMRS y se sube al servidor \emph{(figura \ref{fig:deployopenmrscontexto})}, en este momento existe la posibilidad de modificar el nombre y el contexto de la aplicación. Esto último hace referencia al nombre de la URL bajo el cual funcionará la aplicación OpenMRS que se ha instalado  y el contexto sobre el cual correrá. Si se desea agregar más de 1 OpenMRS al mismo dominio entonces este nombre y contexto deben ser modificados  antes de continuar.\newline


\begin{figure}
 \centering
 \includegraphics[width=15cm]{./images/deploypopenmrs.png}
 % deploypopenmrs.png: 916x491 pixel, 72dpi, 32.31x17.32 cm, bb=0 0 916 491
 \caption{Vista del panel de administración momentos antes de hacer deploy de la aplicación OpenMRS}
 \label{fig:deployopenmrscontexto}
\end{figure}

Se puede verificar que el deploy se realizó correctamente observando el log del proceso.

\newpage 
\subsection{Configurar OpenMRS con MySQL}

Se realiza una conexión a mysql con usuario root desde terminal:

\lstset{language=sh}
\begin{lstlisting}
root@localhost: Mysql -u root -p
\end{lstlisting}

Se crea una base de datos y usuario asociado asignando todos los privilegios.


\lstset{language=SQL}
\begin{lstlisting}
CREATE USER 'MINUEVOOPENMRS'@'localhost' IDENTIFIED BY '***';

GRANT USAGE ON * . * TO 'MINUEVOOPENMRS'@'localhost' IDENTIFIED BY '***' WITH MAX_QUERIES_PER_HOUR 0 MAX_CONNECTIONS_PER_HOUR 0 MAX_UPDATES_PER_HOUR 0 MAX_USER_CONNECTIONS 0 ;

CREATE DATABASE IF NOT EXISTS `MINUEVOOPENMRS` ;

GRANT ALL PRIVILEGES ON `MINUEVOOPENMRS` . * TO 'MINUEVOOPENMRS'@'localhost';
\end{lstlisting}

Entrar por el navegador a la dirección http://miserver.lazos.cl:8080/openmrs y seguir los  pasos para la configuración.

\begin{itemize}
 \item Paso 1: Configuración de la base de datos: Se debe seleccionar la opción ``NO`` y proporcionar datos de usuario y contraseña de MySQL.

  \item Paso 2: Las primeras dos opciones en “SI” luego proporcionar usuario y contraseña de MySQL.

  \item Paso 3: Dejar las opciones por defecto y continuar

  \item Paso 4: Configurar una contraseña para administración, debe contener números y letras mayúsculas y minúsculas.

  \item Paso 5: Configurar una implementación y continuar

  \item Paso 6: Presionar Finish y esperar.
\end{itemize}


Con esto  OpenMRS queda configurado y listo para trabajar.

\subsection{Ingresando a OpenMRS}

Ingresamos desde el navegador a la URL: http://miserver.lazos.cl:8080/openmrs y luego ingresamos usuario y password configurados en la instalación. El navegador muestra la pantalla de la \emph{(figura \ref{fig:pantallainicialopenmrs})}.

\begin{figure}
 \centering
 \includegraphics[width=15cm]{./images/openmrs_pantallainicial.png}
 % openmrs_pantallainicial.png: 1280x800 pixel, 72dpi, 45.16x28.22 cm, bb=0 0 1280 800
 \caption{Vista de la pantalla de ingreso de OpenMRS}
 \label{fig:pantallainicialopenmrs}
\end{figure}

\section{Alfresco}



%\subsection{Sobre Alfresco}

``Alfresco es un sistema de administración de contenidos libre, basado en estándares abiertos y de escala empresarial para sistemas operativos tipo Unix  y Otros. Está diseñado para usuarios que requieren un alto grado de modularidad y rendimiento escalable. Alfresco incluye un repositorio de contenidos, un framework de portal web para administrar y usar contenido estándar en portales, una interfaz CIFS  que provee compatibilidad de sistemas de archivos en Windows y sistemas operativos tipo Unix, un sistema de administración de contenido web capacidad de virtualizar aplicaciones web y sitios estáticos vía Apache Tomcat, búsquedas vía el motor Lucene y flujo de trabajo en jBPM. Alfresco está desarrollado en Java.'' \footnote{\url{http://es.wikipedia.org/wiki/Alfresco}}


\subsection{Obtener Alfresco y pre-requisitos}

\textbf{Obtener} el archivo war de alfresco “alfresco-community-war-3.3.tar.gz" del sitio:\\ \url{http://process.alfresco.com/ccdl/?file=release/community/build-2765/alfresco-community-war-3.3.tar.gz}
    \lstset{language=sh,label=alfresco_obteniendo}
    \begin{lstlisting}
      root@localhost: cd /tmp/ 
      root@localhost: tar vzxf alfresco-community-war-3.3.tar.gz
    \end{lstlisting}

Realizar los siguientes ajustes en MySQL (Debe estar instalado) 
    \lstset{language=sh,label=alfresco_entrandoAMYSQL}
    \begin{lstlisting}
    root@localhost: mysql -u root -p
    \end{lstlisting}

    \lstset{language=SQL,label=alfresco_CreandoBaseDatos}
    \begin{lstlisting}
    create database alfresco;
    create user 'NOMBRE_DE_USUARIO_ALFRESCO'@'localhost' identified by 'PASSWORD_USUARIO_ALFRESCO';
    grant all privileges on alfresco.* to 'NOMBRE_DE_USUARIO_ALFRESCO'@'localhost';
    \end{lstlisting}

\subsection{Configurar}

Configurar Alfresco antes de deplegarlo en glassfish (hacer el deploy).Desde Alfresco 3,3, todos los ajustes ahora se basan en el archivo \textbf{alfresco-global.properties}. Este archivo se encuentra en el package de alfresco del \textbf{alfresco-community-war-3.3.tar.gz}. Este archivo \textbf{alfresco-global.properties} se copia desde la ubicación original hacia la ruta de destino donde glassfish lo reconocerá:

    \lstset{language=sh,label=alfresco_config}
    \begin{lstlisting}
    root@localhost: cp /alfresco-community-war-3.3/extensions/extension/alfresco-global.properties /home/glassfish/LiferayPortal/glassfish/domains/domain1/lib/classes/alfresco-global.properties
    \end{lstlisting}

Modificar este\textbf{ alfresco-global.properties} que se copió en el directorio : \textbf{glassfish/domains/domain1/lib/classes/} para configurar el acceso a la base de datos en alfresco. 
    \lstset{language=sh,label=alfresco_config}
    \begin{lstlisting}
		dir.root = /home/glassfish/LiferayPortal/
		db.username=NOMBRE_DE_USUARIO_ALFRESCO
		db.password=PASSWORD_USUARIO_ALFRESCO
		db.driver=org.gjt.mm.mysql.Driver
		db.url=jdbc:mysql://localhost/alfresco
    \end{lstlisting}

Crear un archivo ``sun-web.xml'' para la configuración de Alfresco.\newline

    \lstset{language=xml,label=sun-web_XML}
    \begin{lstlisting}
      <?xml version="1.0" encoding="UTF-8"?>
      <!DOCTYPE sun-web-app PUBLIC "-//Sun Microsystems, Inc.//DTD Application Server 8.1 Servlet 2.4//EN" "http://www.sun.com/software/appserver/dtds/sun-web-app_2_4-1.dtd">
      <sun-web-app>
      <class-loader delegate="false"/>
      <property name="useMyFaces" value="true"/>
      </sun-web-app>
    \end{lstlisting}

Añadir el archivo ``sun-web.xml''  al archivo war de alfresco (alfresco.war). Para esto vamos al directorio ``\textbf{alfresco-community-war-3.3}'' (en donde se descomprimió el  \textbf{alfresco-community-war-3.3.tar.gz}. Crear aquí el directorio WEB-INF y dejar el sun-web.xml en esta carpeta. Luego se agrega este archivo xml al alfresco.war.\newline

    \lstset{language=sh}
    \begin{lstlisting}
  [root@localhost: tmp]# mkdir -p /alfresco-community-war-3.3/WEB-INF/
  [root@localhost:  tmp]# cp sun-web.xml /alfresco-community-war-3.3/WEB-INF/sun-web.xml
  [root@localhost:  tmp]# cd /alfresco-community-war-3.3/
  [root@localhost:  alfresco-community-war-3.3]# jar -uvf alfresco.war WEB-INF/*.xml
  [root@localhost:  alfresco-community-war-3.3]# chown glassfish:glassfish alfresco.war
  [root@localhost:  alfresco-community-war-3.3]# cp alfresco.war /tmp
    \end{lstlisting}


\subsection{Iniciar}
Iniciar Servidor Glassfishv2.\newline

    \lstset{language=sh}
    \begin{lstlisting}
	[root@localhost:  alfresco-community-war-3.3]# service start glassfish 
	#esto siempre y cuando exista el fichero script en el directorio /etc/init.d/glassfish
    \end{lstlisting}

Hacer el despliegue (deploy) de alfresco.war en el servidor glassfishv2. Para esto iniciamos el servicio de glassfishv2, luego iniciamos el administrador y desde aquí integrar (hacer el deploy) de alfresco.war (que debería estar en /tmp/alfresco-community-war-3.3). \newline

\textbf{Iniciar alfresco: }Puede iniciarse desde el administrador de glassfis (Web aplication - deploy) aquí elegir el 	alfresco.war del directorio.



%\input{bibliografia.tex}



\end{document}
