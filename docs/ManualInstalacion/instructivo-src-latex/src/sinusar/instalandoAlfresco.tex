\subsection{Sobre Alfresco}

``Alfresco es un sistema de administración de contenidos libre, basado en estándares abiertos y de escala empresarial para sistemas operativos tipo Unix  y Otros. Está diseñado para usuarios que requieren un alto grado de modularidad y rendimiento escalable. Alfresco incluye un repositorio de contenidos, un framework de portal web para administrar y usar contenido estándar en portales, una interfaz CIFS  que provee compatibilidad de sistemas de archivos en Windows y sistemas operativos tipo Unix, un sistema de administración de contenido web capacidad de virtualizar aplicaciones web y sitios estáticos vía Apache Tomcat, búsquedas vía el motor Lucene y flujo de trabajo en jBPM. Alfresco está desarrollado en Java.'' \footnote{\url{http://es.wikipedia.org/wiki/Alfresco}}


\subsection{Obtener Alfresco y pre-requisitos}

\textbf{Obtener} el archivo war de alfresco “alfresco-community-war-3.3.tar.gz" del sitio:\\ \url{http://process.alfresco.com/ccdl/?file=release/community/build-2765/alfresco-community-war-3.3.tar.gz}
    \lstset{language=sh,label=alfresco_obteniendo}
    \begin{lstlisting}
      root@localhost: cd /tmp/ 
      root@localhost: tar vzxf alfresco-community-war-3.3.tar.gz
    \end{lstlisting}

Realizar los siguientes ajustes en MySQL (Debe estar instalado) 
    \lstset{language=sh,label=alfresco_entrandoAMYSQL}
    \begin{lstlisting}
    root@localhost: mysql -u root -p
    \end{lstlisting}

    \lstset{language=SQL,label=alfresco_CreandoBaseDatos}
    \begin{lstlisting}
    create database alfresco;
    create user 'NOMBRE_DE_USUARIO_ALFRESCO'@'localhost' identified by 'PASSWORD_USUARIO_ALFRESCO';
    grant all privileges on alfresco.* to 'NOMBRE_DE_USUARIO_ALFRESCO'@'localhost';
    \end{lstlisting}

\subsection{Configurar}

Configurar Alfresco antes de deplegarlo en glassfish (hacer el deploy).Desde Alfresco 3,3, todos los ajustes ahora se basan en el archivo \textbf{alfresco-global.properties}. Este archivo se encuentra en el package de alfresco del \textbf{alfresco-community-war-3.3.tar.gz}. Este archivo \textbf{alfresco-global.properties} se copia desde la ubicación original hacia la ruta de destino donde glassfish lo reconocerá:

    \lstset{language=sh,label=alfresco_config}
    \begin{lstlisting}
    root@localhost: cp /alfresco-community-war-3.3/extensions/extension/alfresco-global.properties /home/glassfish/LiferayPortal/glassfish/domains/domain1/lib/classes/alfresco-global.properties
    \end{lstlisting}

Modificar este\textbf{ alfresco-global.properties} que se copió en el directorio : \textbf{glassfish/domains/domain1/lib/classes/} para configurar el acceso a la base de datos en alfresco. 
    \lstset{language=sh,label=alfresco_config}
    \begin{lstlisting}
		dir.root = /home/glassfish/LiferayPortal/
		db.username=NOMBRE_DE_USUARIO_ALFRESCO
		db.password=PASSWORD_USUARIO_ALFRESCO
		db.driver=org.gjt.mm.mysql.Driver
		db.url=jdbc:mysql://localhost/alfresco
    \end{lstlisting}

Crear un archivo ``sun-web.xml'' para la configuración de Alfresco.\newline

    \lstset{language=xml,label=sun-web_XML}
    \begin{lstlisting}
      <?xml version="1.0" encoding="UTF-8"?>
      <!DOCTYPE sun-web-app PUBLIC "-//Sun Microsystems, Inc.//DTD Application Server 8.1 Servlet 2.4//EN" "http://www.sun.com/software/appserver/dtds/sun-web-app_2_4-1.dtd">
      <sun-web-app>
      <class-loader delegate="false"/>
      <property name="useMyFaces" value="true"/>
      </sun-web-app>
    \end{lstlisting}

Añadir el archivo ``sun-web.xml''  al archivo war de alfresco (alfresco.war). Para esto vamos al directorio ``\textbf{alfresco-community-war-3.3}'' (en donde se descomprimió el  \textbf{alfresco-community-war-3.3.tar.gz}. Crear aquí el directorio WEB-INF y dejar el sun-web.xml en esta carpeta. Luego se agrega este archivo xml al alfresco.war.\newline

    \lstset{language=sh}
    \begin{lstlisting}
  [root@localhost: tmp]# mkdir -p /alfresco-community-war-3.3/WEB-INF/
  [root@localhost:  tmp]# cp sun-web.xml /alfresco-community-war-3.3/WEB-INF/sun-web.xml
  [root@localhost:  tmp]# cd /alfresco-community-war-3.3/
  [root@localhost:  alfresco-community-war-3.3]# jar -uvf alfresco.war WEB-INF/*.xml
  [root@localhost:  alfresco-community-war-3.3]# chown glassfish:glassfish alfresco.war
  [root@localhost:  alfresco-community-war-3.3]# cp alfresco.war /tmp
    \end{lstlisting}


\subsection{Iniciar}
Iniciar Servidor Glassfishv2.\newline

    \lstset{language=sh}
    \begin{lstlisting}
	[root@localhost:  alfresco-community-war-3.3]# service start glassfish 
	#esto siempre y cuando exista el fichero script en el directorio /etc/init.d/glassfish
    \end{lstlisting}

Hacer el despliegue (deploy) de alfresco.war en el servidor glassfishv2. Para esto iniciamos el servicio de glassfishv2, luego iniciamos el administrador y desde aquí integrar (hacer el deploy) de alfresco.war (que debería estar en /tmp/alfresco-community-war-3.3). \newline

\textbf{Iniciar alfresco: }Puede iniciarse desde el administrador de glassfis (Web aplication - deploy) aquí elegir el 	alfresco.war del directorio.


